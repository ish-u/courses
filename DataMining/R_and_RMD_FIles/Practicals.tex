% Options for packages loaded elsewhere
\PassOptionsToPackage{unicode}{hyperref}
\PassOptionsToPackage{hyphens}{url}
%
\documentclass[
]{article}
\title{R Notebook}
\author{}
\date{\vspace{-2.5em}}

\usepackage{amsmath,amssymb}
\usepackage{lmodern}
\usepackage{iftex}
\ifPDFTeX
  \usepackage[T1]{fontenc}
  \usepackage[utf8]{inputenc}
  \usepackage{textcomp} % provide euro and other symbols
\else % if luatex or xetex
  \usepackage{unicode-math}
  \defaultfontfeatures{Scale=MatchLowercase}
  \defaultfontfeatures[\rmfamily]{Ligatures=TeX,Scale=1}
\fi
% Use upquote if available, for straight quotes in verbatim environments
\IfFileExists{upquote.sty}{\usepackage{upquote}}{}
\IfFileExists{microtype.sty}{% use microtype if available
  \usepackage[]{microtype}
  \UseMicrotypeSet[protrusion]{basicmath} % disable protrusion for tt fonts
}{}
\makeatletter
\@ifundefined{KOMAClassName}{% if non-KOMA class
  \IfFileExists{parskip.sty}{%
    \usepackage{parskip}
  }{% else
    \setlength{\parindent}{0pt}
    \setlength{\parskip}{6pt plus 2pt minus 1pt}}
}{% if KOMA class
  \KOMAoptions{parskip=half}}
\makeatother
\usepackage{xcolor}
\IfFileExists{xurl.sty}{\usepackage{xurl}}{} % add URL line breaks if available
\IfFileExists{bookmark.sty}{\usepackage{bookmark}}{\usepackage{hyperref}}
\hypersetup{
  pdftitle={R Notebook},
  hidelinks,
  pdfcreator={LaTeX via pandoc}}
\urlstyle{same} % disable monospaced font for URLs
\usepackage[margin=1in]{geometry}
\usepackage{color}
\usepackage{fancyvrb}
\newcommand{\VerbBar}{|}
\newcommand{\VERB}{\Verb[commandchars=\\\{\}]}
\DefineVerbatimEnvironment{Highlighting}{Verbatim}{commandchars=\\\{\}}
% Add ',fontsize=\small' for more characters per line
\usepackage{framed}
\definecolor{shadecolor}{RGB}{248,248,248}
\newenvironment{Shaded}{\begin{snugshade}}{\end{snugshade}}
\newcommand{\AlertTok}[1]{\textcolor[rgb]{0.94,0.16,0.16}{#1}}
\newcommand{\AnnotationTok}[1]{\textcolor[rgb]{0.56,0.35,0.01}{\textbf{\textit{#1}}}}
\newcommand{\AttributeTok}[1]{\textcolor[rgb]{0.77,0.63,0.00}{#1}}
\newcommand{\BaseNTok}[1]{\textcolor[rgb]{0.00,0.00,0.81}{#1}}
\newcommand{\BuiltInTok}[1]{#1}
\newcommand{\CharTok}[1]{\textcolor[rgb]{0.31,0.60,0.02}{#1}}
\newcommand{\CommentTok}[1]{\textcolor[rgb]{0.56,0.35,0.01}{\textit{#1}}}
\newcommand{\CommentVarTok}[1]{\textcolor[rgb]{0.56,0.35,0.01}{\textbf{\textit{#1}}}}
\newcommand{\ConstantTok}[1]{\textcolor[rgb]{0.00,0.00,0.00}{#1}}
\newcommand{\ControlFlowTok}[1]{\textcolor[rgb]{0.13,0.29,0.53}{\textbf{#1}}}
\newcommand{\DataTypeTok}[1]{\textcolor[rgb]{0.13,0.29,0.53}{#1}}
\newcommand{\DecValTok}[1]{\textcolor[rgb]{0.00,0.00,0.81}{#1}}
\newcommand{\DocumentationTok}[1]{\textcolor[rgb]{0.56,0.35,0.01}{\textbf{\textit{#1}}}}
\newcommand{\ErrorTok}[1]{\textcolor[rgb]{0.64,0.00,0.00}{\textbf{#1}}}
\newcommand{\ExtensionTok}[1]{#1}
\newcommand{\FloatTok}[1]{\textcolor[rgb]{0.00,0.00,0.81}{#1}}
\newcommand{\FunctionTok}[1]{\textcolor[rgb]{0.00,0.00,0.00}{#1}}
\newcommand{\ImportTok}[1]{#1}
\newcommand{\InformationTok}[1]{\textcolor[rgb]{0.56,0.35,0.01}{\textbf{\textit{#1}}}}
\newcommand{\KeywordTok}[1]{\textcolor[rgb]{0.13,0.29,0.53}{\textbf{#1}}}
\newcommand{\NormalTok}[1]{#1}
\newcommand{\OperatorTok}[1]{\textcolor[rgb]{0.81,0.36,0.00}{\textbf{#1}}}
\newcommand{\OtherTok}[1]{\textcolor[rgb]{0.56,0.35,0.01}{#1}}
\newcommand{\PreprocessorTok}[1]{\textcolor[rgb]{0.56,0.35,0.01}{\textit{#1}}}
\newcommand{\RegionMarkerTok}[1]{#1}
\newcommand{\SpecialCharTok}[1]{\textcolor[rgb]{0.00,0.00,0.00}{#1}}
\newcommand{\SpecialStringTok}[1]{\textcolor[rgb]{0.31,0.60,0.02}{#1}}
\newcommand{\StringTok}[1]{\textcolor[rgb]{0.31,0.60,0.02}{#1}}
\newcommand{\VariableTok}[1]{\textcolor[rgb]{0.00,0.00,0.00}{#1}}
\newcommand{\VerbatimStringTok}[1]{\textcolor[rgb]{0.31,0.60,0.02}{#1}}
\newcommand{\WarningTok}[1]{\textcolor[rgb]{0.56,0.35,0.01}{\textbf{\textit{#1}}}}
\usepackage{graphicx}
\makeatletter
\def\maxwidth{\ifdim\Gin@nat@width>\linewidth\linewidth\else\Gin@nat@width\fi}
\def\maxheight{\ifdim\Gin@nat@height>\textheight\textheight\else\Gin@nat@height\fi}
\makeatother
% Scale images if necessary, so that they will not overflow the page
% margins by default, and it is still possible to overwrite the defaults
% using explicit options in \includegraphics[width, height, ...]{}
\setkeys{Gin}{width=\maxwidth,height=\maxheight,keepaspectratio}
% Set default figure placement to htbp
\makeatletter
\def\fps@figure{htbp}
\makeatother
\setlength{\emergencystretch}{3em} % prevent overfull lines
\providecommand{\tightlist}{%
  \setlength{\itemsep}{0pt}\setlength{\parskip}{0pt}}
\setcounter{secnumdepth}{-\maxdimen} % remove section numbering
\ifLuaTeX
  \usepackage{selnolig}  % disable illegal ligatures
\fi

\begin{document}
\maketitle

\begin{Shaded}
\begin{Highlighting}[]
\NormalTok{peoplefile }\OtherTok{\textless{}{-}} \FunctionTok{read.delim}\NormalTok{(}\StringTok{"D:/Data Mining/Practicals/people.txt"}\NormalTok{,}\AttributeTok{sep=}\StringTok{" "}\NormalTok{)}
\NormalTok{peoplefile}
\end{Highlighting}
\end{Shaded}

\begin{verbatim}
##   Age agegroup height  status yearsmarried
## 1  21    adult    6.0  single           -1
## 2   2    child    3.0 married            0
## 3  18    adult    5.7 married           20
## 4 221  elderly    5.0 widowed            2
## 5  34    child   -7.0 married            3
\end{verbatim}

\begin{Shaded}
\begin{Highlighting}[]
\FunctionTok{library}\NormalTok{(sqldf)}
\end{Highlighting}
\end{Shaded}

\begin{verbatim}
## Loading required package: gsubfn
\end{verbatim}

\begin{verbatim}
## Loading required package: proto
\end{verbatim}

\begin{verbatim}
## Loading required package: RSQLite
\end{verbatim}

\begin{Shaded}
\begin{Highlighting}[]
\NormalTok{totalRows }\OtherTok{\textless{}{-}} \FunctionTok{nrow}\NormalTok{(peoplefile)}
\FunctionTok{paste}\NormalTok{(totalRows)}
\end{Highlighting}
\end{Shaded}

\begin{verbatim}
## [1] "5"
\end{verbatim}

\begin{Shaded}
\begin{Highlighting}[]
\NormalTok{query1 }\OtherTok{\textless{}{-}} \StringTok{"SELECT}
\StringTok{                            AGE,}
\StringTok{                            AGEGROUP,}
\StringTok{                            HEIGHT,}
\StringTok{                            STATUS,}
\StringTok{                            YEARSMARRIED}
\StringTok{                        FROM}
\StringTok{                            peoplefile}
\StringTok{                        WHERE}
\StringTok{                            AGE \textgreater{}=0 AND AGE \textless{}=150"}
\FunctionTok{sqldf}\NormalTok{(query1)}
\end{Highlighting}
\end{Shaded}

\begin{verbatim}
##   Age agegroup height  status yearsmarried
## 1  21    adult    6.0  single           -1
## 2   2    child    3.0 married            0
## 3  18    adult    5.7 married           20
## 4  34    child   -7.0 married            3
\end{verbatim}

\begin{Shaded}
\begin{Highlighting}[]
\NormalTok{rule1Voilated }\OtherTok{\textless{}{-}}\NormalTok{ totalRows }\SpecialCharTok{{-}} \FunctionTok{nrow}\NormalTok{(}\FunctionTok{sqldf}\NormalTok{(query1))}
\end{Highlighting}
\end{Shaded}

\begin{Shaded}
\begin{Highlighting}[]
\CommentTok{\#RULE SET 2}
\NormalTok{query2 }\OtherTok{\textless{}{-}} \StringTok{"SELECT}
\StringTok{                    AGE,}
\StringTok{                    AGEGROUP,}
\StringTok{                    HEIGHT,}
\StringTok{                    STATUS,}
\StringTok{                    YEARSMARRIED}
\StringTok{                FROM}
\StringTok{                    peoplefile}
\StringTok{                WHERE}
\StringTok{                    AGE \textgreater{} YEARSMARRIED"}
\FunctionTok{sqldf}\NormalTok{(query2)}
\end{Highlighting}
\end{Shaded}

\begin{verbatim}
##   Age agegroup height  status yearsmarried
## 1  21    adult      6  single           -1
## 2   2    child      3 married            0
## 3 221  elderly      5 widowed            2
## 4  34    child     -7 married            3
\end{verbatim}

\begin{Shaded}
\begin{Highlighting}[]
\NormalTok{rule2Voilated }\OtherTok{\textless{}{-}}\NormalTok{ totalRows }\SpecialCharTok{{-}} \FunctionTok{nrow}\NormalTok{(}\FunctionTok{sqldf}\NormalTok{(query2))}
\end{Highlighting}
\end{Shaded}

\begin{Shaded}
\begin{Highlighting}[]
\CommentTok{\#RULE SET 3}
\NormalTok{query3 }\OtherTok{\textless{}{-}} \StringTok{"SELECT}
\StringTok{                  AGE,}
\StringTok{                  AGEGROUP,}
\StringTok{                  HEIGHT,}
\StringTok{                  STATUS,}
\StringTok{                  YEARSMARRIED}
\StringTok{              FROM}
\StringTok{                  peoplefile}
\StringTok{              WHERE}
\StringTok{                      STATUS=\textquotesingle{}married\textquotesingle{} OR STATUS=\textquotesingle{}single\textquotesingle{}OR STATUS=\textquotesingle{}widowed\textquotesingle{} "}
\FunctionTok{sqldf}\NormalTok{(query3)}
\end{Highlighting}
\end{Shaded}

\begin{verbatim}
##   Age agegroup height  status yearsmarried
## 1  21    adult    6.0  single           -1
## 2   2    child    3.0 married            0
## 3  18    adult    5.7 married           20
## 4 221  elderly    5.0 widowed            2
## 5  34    child   -7.0 married            3
\end{verbatim}

\begin{Shaded}
\begin{Highlighting}[]
\NormalTok{rule3Voilated }\OtherTok{\textless{}{-}}\NormalTok{ totalRows }\SpecialCharTok{{-}} \FunctionTok{nrow}\NormalTok{(}\FunctionTok{sqldf}\NormalTok{(query3))}
\end{Highlighting}
\end{Shaded}

\begin{Shaded}
\begin{Highlighting}[]
\CommentTok{\#RULE SET 4}
\NormalTok{query4 }\OtherTok{\textless{}{-}} \StringTok{"SELECT}
\StringTok{                AGE,}
\StringTok{                AGEGROUP,}
\StringTok{                HEIGHT,}
\StringTok{                STATUS,}
\StringTok{                YEARSMARRIED}
\StringTok{          FROM}
\StringTok{                peoplefile}
\StringTok{          WHERE}
\StringTok{                AGE \textless{} 18 AND AGEGROUP = \textquotesingle{}child\textquotesingle{} OR}
\StringTok{                AGE BETWEEN 18 AND 65 AND AGEGROUP = \textquotesingle{}adult\textquotesingle{} OR }
\StringTok{                AGE \textgreater{} 65 AND AGEGROUP = \textquotesingle{}elderly\textquotesingle{} "}
\FunctionTok{sqldf}\NormalTok{(query4)}
\end{Highlighting}
\end{Shaded}

\begin{verbatim}
##   Age agegroup height  status yearsmarried
## 1  21    adult    6.0  single           -1
## 2   2    child    3.0 married            0
## 3  18    adult    5.7 married           20
## 4 221  elderly    5.0 widowed            2
\end{verbatim}

\begin{Shaded}
\begin{Highlighting}[]
\NormalTok{rule4Voilated }\OtherTok{\textless{}{-}}\NormalTok{ totalRows }\SpecialCharTok{{-}} \FunctionTok{nrow}\NormalTok{(}\FunctionTok{sqldf}\NormalTok{(query4))}
\end{Highlighting}
\end{Shaded}

\begin{Shaded}
\begin{Highlighting}[]
\FunctionTok{cat}\NormalTok{(}\StringTok{"Rule Set Voilations Summary }\SpecialCharTok{\textbackslash{}n}\StringTok{"}\NormalTok{,}
    \StringTok{"RuleSet 1 :"}\NormalTok{,rule1Voilated,}\StringTok{"}\SpecialCharTok{\textbackslash{}n}\StringTok{"}\NormalTok{,}
    \StringTok{"RuleSet 2 :"}\NormalTok{,rule2Voilated,}\StringTok{"}\SpecialCharTok{\textbackslash{}n}\StringTok{"}\NormalTok{,}
    \StringTok{"RuleSet 3 :"}\NormalTok{,rule3Voilated,}\StringTok{"}\SpecialCharTok{\textbackslash{}n}\StringTok{"}\NormalTok{,}
    \StringTok{"RuleSet 4 :"}\NormalTok{,rule4Voilated,}\StringTok{"}\SpecialCharTok{\textbackslash{}n}\StringTok{"}\NormalTok{)}
\end{Highlighting}
\end{Shaded}

\begin{verbatim}
## Rule Set Voilations Summary 
##  RuleSet 1 : 1 
##  RuleSet 2 : 1 
##  RuleSet 3 : 0 
##  RuleSet 4 : 1
\end{verbatim}

\begin{Shaded}
\begin{Highlighting}[]
\NormalTok{totalVoilations }\OtherTok{\textless{}{-}} \FunctionTok{as.integer}\NormalTok{(}\FunctionTok{c}\NormalTok{(rule1Voilated,rule2Voilated,rule3Voilated,rule4Voilated))}
\FunctionTok{plot}\NormalTok{(totalVoilations,}\AttributeTok{xlab =} \StringTok{"Rule"}\NormalTok{,}\AttributeTok{ylab =} \StringTok{"Number of Times Violated"}\NormalTok{)}
\end{Highlighting}
\end{Shaded}

\includegraphics{Practicals_files/figure-latex/unnamed-chunk-9-1.pdf}

\begin{Shaded}
\begin{Highlighting}[]
\FunctionTok{barplot}\NormalTok{(totalVoilations,}\AttributeTok{xlab=}\StringTok{"Rules"}\NormalTok{,}\AttributeTok{ylab =} \StringTok{"Number of Times Violated"}\NormalTok{,}\AttributeTok{names.arg =} \FunctionTok{c}\NormalTok{(}\StringTok{"Rule1"}\NormalTok{,}\StringTok{"Rule 2"}\NormalTok{,}\StringTok{"Rule 3"}\NormalTok{,}\StringTok{"Rule 4"}\NormalTok{))}
\end{Highlighting}
\end{Shaded}

\includegraphics{Practicals_files/figure-latex/unnamed-chunk-9-2.pdf}

\end{document}
